\documentclass{resume}

\usepackage{hyperref}

\name{Minsu Sun}

\email{\href{mailto:poodding397@gmail.com}{poodding397 [at] gmail [dot] com}}

\begin{document}

\begin{rSection}{Education}
    \begin{rSubsection}{POSTECH (Pohang University of Science and Technology)}{Feb 2022 -- (Expected)Feb 2024}
        - Majoring in \href{https://cse.postech.ac.kr}{Computer Science and Engineering} \\
        - Awarded CSE Global Leadership Program scholarship(Sep 2023)
    \end{rSubsection}
\end{rSection}

\begin{rSection}{Skills}
    \begin{tabular}{@{}p{0.25\linewidth}p{0.7\linewidth}}
        Programming Languages
            & C/C++, C\#, Python, Java \\ [0.3em]

        Frameworks
            & Flask, FastAPI, BeautifulSoup4, Selenium \\ [0.3em]

        Tools
            & Git/Github, Docker, AWS, Kubernetes, Helm \\ [0.3em]

        Natural Languages
            & Korean(Native), English(Intermediate)
    \end{tabular}
\end{rSection}

\begin{rSection}{Job Experience}
    \begin{rSubsection}{\href{https://shiftup.co.kr/}{SHIFT UP}}{Jun 2024 -- Aug 2024}
        {\bfseries\rmfamily{Intern Backend Engineer of \href{https://nikke-en.com/}{NIKKE}}}

        \item Worked on:

        \vspace{-0.5em}
        \begin{list}{--}{}
            \itemsep -0.5em

            \item Implementing features of collaborate event mini game

            \item Deploying and managing infra
        \end{list}

        \item Skills: C\#, Kubernetes, Helm
    \end{rSubsection}

    \begin{rSubsection}{\href{https://unitcompany.co.kr}{UnitCompany Inc.}}{Jun 2022 -- Jun 2024}
        {\bfseries\rmfamily{Part-time Backend Engineer}}

        \item Worked on educational service backend development

        \item Skills: Python, AWS, Docker
    \end{rSubsection}

    \begin{rSubsection}{\href{https://unitcompany.co.kr}{UnitCompany Inc.}}{May 2022 -- Jun 2024}
        {\bfseries\rmfamily{Intern Researcher}}

        \item Worked on research about recommendation system and LLM

        \item Topics:

        \vspace{-0.5em}
        \begin{list}{--}{}
            \itemsep -0.5em

            \item BERT based contents recommendation system

            \item LLM Fine-Tuning

            \item LLM Prompt Engineering
        \end{list}
    \end{rSubsection}
\end{rSection}

\begin{rSection}{Projects}
    \begin{rSubsection}{Distributed Arduino Calculator}{Aug 2024}
        Simple Distributed Computing Cluster with Arduinos via $I^2C$ protocol

        \item Individual project simulating distributed 32bit floating point calculation(\href{https://github.com/minsusun/kakashi/tree/main/avr/arduino/i2c-distributed-floating-point-calculator}{link})

        \item Distributed 32bit floating point(fp32) addition operations with 4 workers(Arduino Uno)

        \item Approximately took 10 seconds on 400K fp32 addition operations
    \end{rSubsection}

    \begin{rSubsection}{Sponge}{Mar 2024 -- Jun 2024}
        Educational TCP/IP Development Project Sponge

        \item Individual project implementing TCP/IP stack on Linux(\href{https://github.com/minsusun/csed353-sponge}{link})

        \item Skills: C++

        \item Additionally implemented SHA256 hash calculation of every frame for debugging and verify purpose
    \end{rSubsection}

    \begin{rSubsection}{BLARE}{Mar 2024 -- Jun 2024}
        Blended FLARE(Forward-Looking Active Retrieval Augmented Generation)

        \item Individual research project proposing and implementing blended query formulation method in FLARE(\href{https://github.com/minsusun/BLARE}{link})

        \item Topics: NLP, LLM, RAG(Retreival Augmented Generation)
    \end{rSubsection}

    \begin{rSubsection}{RISC-V CPU}{Mar 2024 -- Jun 2024}
        RISC-V 5-Stage Pipelined CPU with Configurable Cache

        \item Team project implementing RISC-V 5-Stage Pipelined CPU with Verilog(\href{https://github.com/minsusun/csed311/tree/main/lab5}{link})

        \item Features:
        \vspace{-0.5em}
        \begin{list}{--}{\setlength{\rightmargin}{1.5em}}
            \itemsep -0.5em

            \item Pipelined 5-Stage Execution

            \item 2-Bit Saturation Counter Branch Predictor with PHT(Prediction History Table) and BTB(Branch Target Buffer)

            \item Multi-way configurable cache based on LRU
        \end{list}
    \end{rSubsection}

    \begin{rSubsection}{CUDA Based Parallel KNN Calculation}{Dec 2023}
        Optimizing naive KNN(K Nearest Neighbors) operation executed on CUDA

        \item Individual research project optimizing naive KNN operation on CUDA device

        \item Composed KNN operation as a combination of belows
        \vspace{-0.5em}
        \begin{list}{--}{\setlength{\rightmargin}{1.5em}}
            \itemsep -0.5em

            \item Build euclidean distance matrix of given points in the manner of matrix multiplication using tiling

            \item Sort distances of neighbors using thrust::sort and select K neighbors
        \end{list}

        \item Achievement compared to naive baseline code:
        \vspace{-0.5em}
        \begin{list}{--}{\setlength{\rightmargin}{1.5em}}
            \itemsep -0.5em

            \item Calculating distances: about 28 times faster execution time

            \item Sorting distances: about 11 times faster execution time
        \end{list}
    \end{rSubsection}

    \begin{rSubsection}{B-CARAFE}{Nov 2023 -- Dec 2023}
        Better CARAFE(Content-Aware ReAssembly of FEatures)

        \item Individual research project proposing and implementing better reassembly methods

        \item Proposed new reassembly methods with activation functions attached on original reassembly module

        \item Achievement:
        \vspace{-0.5em}
        \begin{list}{--}{\setlength{\rightmargin}{1.5em}}
            \itemsep -0.5em

            \item Proposed B-CARAFE(Faster R-CNN + GELU, ResNet-50) - AP: 23.7 fps: 13.25

            \item Original CARAFE++(Faster R-CNN, ResNet-50) - AP: 22.5 fps: 12.54
        \end{list}

        \item Full paper about the research(\href{https://github.com/minsusun/csed539/blob/main/main.pdf}{link})

        \item Topics: Computer Vision, Image Segmentation, CARAFE(Content-Aware ReAssembly of FEatures)
    \end{rSubsection}

    \begin{rSubsection}{MDEditor}{Oct 2023 -- Dec 2023}
        IntelliJ Real-time Markdown Editor Plugin

        \item Team project developing IntelliJ plugin(\href{https://github.com/minsusun/csed332-project}{link})

        \item Developed based on Agile Software Development and Test Driven-Development

        \item Skills: Java, Git/GitHub

        \item Main Role: Developer, QA
    \end{rSubsection}
    
    \begin{rSubsection}{PintOS}{Sep 2023 -- Dec 2023}
        - Individual project developing educational OS called PintOS(\href{https://github.com/minsusun/csed312-pintos}{link}) \\
        - Skills: C \\
        - Worked on Threading, User Program, Virtual Memory
    \end{rSubsection}
    
    \begin{rSubsection}{BaroKey}{Oct 2023}
        - Team project developing web service(\href{https://github.com/UniD3-Hackathon-Team4/barokey}{link}) introduced at 3rd UniThon Hackathon Track \\
        - Web service supplying user real-time emergency-related issue keywords near user's location \\
        - Skills: Python(FastAPI, BeautifulSoup4, Selenium), AWS EC2 \\
        - Main Role: Backend Developer
    \end{rSubsection}

    \begin{rSubsection}{Arduino MIDI Controller}{May 2023 -- Jun 2023}
        Arduino MIDI Controller for Musical Keyboard

        \item Individual project implementing MIDI controller for musical keyboard

        \item MIDI Controller based on Arduino Leonardo with shift registers and matrix-ed switches
    \end{rSubsection}

    \begin{rSubsection}{RISC-V SRNPU}{Mar 2023 -- Jun 2023}
        RISC-V Based Super Resolution Neural Processing Unit

        \item Individual project implementing SRNPU with Verilog

        \item Hardware accelerator dedicated to generate super resolution image based on CNN model

        \item Processed 128x128 image under 4ms with 3 layers Sim-ESPCN CNN model

        \item Processed 128x128 image under 10ms with 8 layers SSAI 2021 CNN model
    \end{rSubsection}

    \begin{rSubsection}{Online Judge Backend}{Jul 2022}
        - Individual project supplying online judge system application to a company \\
        - Online judge system backend based on Qingdao University's seccomp judger library \\
        - Skills: Python, Docker, AWS SQS
    \end{rSubsection}
    
\end{rSection}

\end{document}

